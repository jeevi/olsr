%
% API Documentation for API Documentation
% Module olsr
%
% Generated by epydoc 3.0.1
% [Fri May 11 18:57:19 2012]
%

%%%%%%%%%%%%%%%%%%%%%%%%%%%%%%%%%%%%%%%%%%%%%%%%%%%%%%%%%%%%%%%%%%%%%%%%%%%
%%                          Module Description                           %%
%%%%%%%%%%%%%%%%%%%%%%%%%%%%%%%%%%%%%%%%%%%%%%%%%%%%%%%%%%%%%%%%%%%%%%%%%%%

    \index{olsr \textit{(module)}|(}
\section{Module olsr}

    \label{olsr}
\begin{alltt}

This is the API file for my OLSR protocol implementation. The methods defined in this file are as follows:

def all\_indices(value, qlist):

     This method takes the arguments value and a list qlist. it returns the list containing the indices of the 'value' in qlist.
    
def receive(addr, isMPR, i, mypeer, source):

     This method takes the arguments, 
    1. addr: The address to which the calling process binds itself to.
    2. isMPR: It is a boolean value. It is True when the calling process [node] is a Multi Point Relay of the source, else it is a two-hop neighbor or a one-hop neighbor which was not selected as an MPR of the source.
    3. i: Unique Identifier of the calling process[node] a.k.a receiver
    4. mypeer: It is a list of one-hop neighbors of the calling process[node].
    5. source: It is a boolean value, True only for the source node.
    
def sendd(mesg, isMPR, worker\_addr, me):

     This method takes the arguments, 
    1. mesg: It is a string, representing the message to be sent.
    2. isMPR: It is a boolean value. It is True when the calling process [node] is a Multi Point Relay of the source, else it is a two-hop neighbor or  a one-hop neighbor which was not selected as an MPR of the source.
    3. worker\_addr: It is the receiver[IP+Port number] to whom the message is to be sent.
    4. me: Unique Identifier of the calling process[node] a.k.a sender

def GetMPR(allNeighbor, thr, i):

     This method takes the arguments, 
    1. allNeighbor: It is a 2-D list, where each 1-D list in this 2-D list represents the one-hop neighbors of the on-hop neighbors of source node.
        e.g. allNeighbor[3] is a 1-D list of one-hop neighbor of the 4th[only in the list] one-hop neighbor of source.
    2. thr: It is the number of processes [nodes].
    3. i: the Identifier of the node whose Multi Point Relays are to be found.


zmq - zeromq, The socket library that provides a API framework for sending messages in IPC. for more information goto - http://www.zeromq.org/
\end{alltt}


%%%%%%%%%%%%%%%%%%%%%%%%%%%%%%%%%%%%%%%%%%%%%%%%%%%%%%%%%%%%%%%%%%%%%%%%%%%
%%                               Functions                               %%
%%%%%%%%%%%%%%%%%%%%%%%%%%%%%%%%%%%%%%%%%%%%%%%%%%%%%%%%%%%%%%%%%%%%%%%%%%%

  \subsection{Functions}

    \label{olsr:all_indices}
    \index{olsr \textit{(module)}!olsr.all\_indices \textit{(function)}}

    \vspace{0.5ex}

\hspace{.8\funcindent}\begin{boxedminipage}{\funcwidth}

    \raggedright \textbf{all\_indices}(\textit{value}, \textit{qlist})

\setlength{\parskip}{2ex}
\setlength{\parskip}{1ex}
    \end{boxedminipage}

    \label{olsr:receive}
    \index{olsr \textit{(module)}!olsr.receive \textit{(function)}}

    \vspace{0.5ex}

\hspace{.8\funcindent}\begin{boxedminipage}{\funcwidth}

    \raggedright \textbf{receive}(\textit{addr}, \textit{isMPR}, \textit{i}, \textit{mypeer}, \textit{source})

\setlength{\parskip}{2ex}
\setlength{\parskip}{1ex}
    \end{boxedminipage}

    \label{olsr:sendd}
    \index{olsr \textit{(module)}!olsr.sendd \textit{(function)}}

    \vspace{0.5ex}

\hspace{.8\funcindent}\begin{boxedminipage}{\funcwidth}

    \raggedright \textbf{sendd}(\textit{mesg}, \textit{isMPR}, \textit{worker\_addr}, \textit{me})

\setlength{\parskip}{2ex}
\setlength{\parskip}{1ex}
    \end{boxedminipage}

    \label{olsr:GetMPR}
    \index{olsr \textit{(module)}!olsr.GetMPR \textit{(function)}}

    \vspace{0.5ex}

\hspace{.8\funcindent}\begin{boxedminipage}{\funcwidth}

    \raggedright \textbf{GetMPR}(\textit{allNeighbor}, \textit{thr}, \textit{i})

\setlength{\parskip}{2ex}
\setlength{\parskip}{1ex}
    \end{boxedminipage}


%%%%%%%%%%%%%%%%%%%%%%%%%%%%%%%%%%%%%%%%%%%%%%%%%%%%%%%%%%%%%%%%%%%%%%%%%%%
%%                               Variables                               %%
%%%%%%%%%%%%%%%%%%%%%%%%%%%%%%%%%%%%%%%%%%%%%%%%%%%%%%%%%%%%%%%%%%%%%%%%%%%

  \subsection{Variables}

    \vspace{-1cm}
\hspace{\varindent}\begin{longtable}{|p{\varnamewidth}|p{\vardescrwidth}|l}
\cline{1-2}
\cline{1-2} \centering \textbf{Name} & \centering \textbf{Description}& \\
\cline{1-2}
\endhead\cline{1-2}\multicolumn{3}{r}{\small\textit{continued on next page}}\\\endfoot\cline{1-2}
\endlastfoot\raggedright \_\-\_\-p\-a\-c\-k\-a\-g\-e\-\_\-\_\- & \raggedright \textbf{Value:} 
{\tt None}&\\
\cline{1-2}
\end{longtable}

    \index{olsr \textit{(module)}|)}
